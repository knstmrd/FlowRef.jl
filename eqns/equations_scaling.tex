\documentclass[a4paper,11pt,english]{article}
\usepackage[verbose=true,a4paper]{geometry}
\AtBeginDocument{
  \newgeometry{
    textheight=9in,
    textwidth=6.5in,
    top=1in,
    headheight=14pt,
    headsep=25pt,
    footskip=30pt
  }
}
\usepackage{amsmath,amssymb,amsfonts}%


\title{Equation Scaling}

\begin{document}

\section{Flow variables and units (SI)}

Flow variables and their units:
\begin{enumerate}
    \item Density $\rho$ is kg/m$^3$
    \item Pressure $p$ is Pa (is N/m$^2$ is J/m$^3$)
    \item Flow velocity $v$ is m/s
    \item Total energy per unit volume $E=\rho e=\rho \varepsilon_{\mathrm{int}} + \rho v^2 / 2$ is J/m$^3$
    \item Total specific energy $e$ is J/kg
    \item Total specific internal energy $\varepsilon$ is J/kg
    \item Mass of the constituent species $m$ is kg
\end{enumerate}

\section{Euler equations}
For a two-dimensional flow of a single-species inviscid gas the compressible Euler equations governing such a flow are given by
\begin{equation}
    \frac{\partial}{\partial t}\mathbf{u} + \frac{\partial}{\partial x}\mathbf{f}_x + \frac{\partial}{\partial y}\mathbf{f}_y = 0.
\end{equation}

The vector of conservative variables  $\mathbf{u} \in \mathbb{R}^4$ is given by
\begin{equation}
     \mathbf{u} = \left(\rho, \rho v_x, \rho v_y, E \right)^{\mathrm{T}},
\end{equation}

The inviscid fluxes are given by
\begin{equation}
     \mathbf{f}_x = \left(\rho v_x, \rho v_x^2 + p, \rho v_x v_y, (E+p)v_x \right)^{\mathrm{T}},
\end{equation}
\begin{equation}
     \mathbf{f}_y = \left(\rho v_y, \rho v_x v_y, \rho v_y^2 + p, (E+p)v_y \right)^{\mathrm{T}}.
\end{equation}
Instead of $E$ one can also write $\rho e$.

Re-writing via scaled variables and reference quantities, we obtain
\begin{equation}
    \frac{\partial}{t_{ref}  \partial \hat{t}}\mathbf{u} + \frac{\partial}{L_{ref} \partial \hat{x}}\mathbf{f}_x + \frac{\partial}{L_{ref}\partial  \hat{y}}\mathbf{f}_y = 0,
\end{equation}
\begin{equation}
     \mathbf{u} = \rho_{ref}\left( \hat{\rho},
     v_{ref} \hat{\rho} \hat{v}_x,
     v_{ref} \hat{\rho} \hat{v}_y,
     p_{ref} \hat{\rho}\hat{e} \right),
\end{equation}
\begin{equation}
     \mathbf{f}_x = \left(\rho_{ref} v_{ref}  \hat{\rho} \hat{v}_x,
     p_{ref} \hat{\rho} \hat{v}_x^2 + p_{ref}\hat{p},
     p_{ref} \hat{\rho} \hat{v}_x \hat{v}_y,
     v_{ref}  p_{ref}(\hat{\rho}\hat{e}+\hat{p})\hat{v}_x \right),
\end{equation}
\begin{equation}
     \mathbf{f}_y = \left(\rho_{ref} v_{ref} \hat{\rho} \hat{v}_y,
     p_{ref} \hat{\rho} \hat{v}_x \hat{v}_y,
     p_{ref} \hat{\rho} \hat{v}_y^2 + p_{ref}\hat{p}, 
     v_{ref} p_{ref}(\hat{\rho}\hat{e}+\hat{p})\hat{v}_y \right).
\end{equation}

Accounting for the fact that $t_{ref}=L_{ref}/v_{ref}$, we can write:
\begin{equation}
    \frac{\partial}{\partial \hat{t}}\hat{\mathbf{u}} + \frac{\partial}{\partial \hat{x}}\hat{\mathbf{f}}_x + \frac{\partial}{\partial  \hat{y}}\hat{\mathbf{f}}_y = 0,
\end{equation}
\begin{equation}
     \mathbf{u} = \left(\hat{\rho},
     \hat{\rho} \hat{v}_x,
     \hat{\rho} \hat{v}_y,
     \hat{\rho}\hat{e},
     \right),
\end{equation}
\begin{equation}
     \mathbf{f}_x = \left(\hat{\rho} \hat{v}_x,
     \hat{\rho} \hat{v}_x^2 + \hat{p},
     \hat{\rho} \hat{v}_x \hat{v}_y,
     (\hat{\rho}\hat{e} + \hat{p})\hat{v}_x
     \right),
\end{equation}
\begin{equation}
     \mathbf{f}_y = \left(\hat{\rho} \hat{v}_y,
     \hat{\rho} \hat{v}_x \hat{v}_y,
     \hat{\rho} \hat{v}_y^2 + \hat{p}, 
     (\hat{\rho}\hat{e}+\hat{p})\hat{v}_y
     \right).
\end{equation}
So the scaled Euler equations are identical to the non-scaled ones, as all reference quantities cancel out.

\subsection{Scaling of specific heats}
We have (for a single-component flow) $T=T_{ref} \hat{T} = m_{ref} p_{ref} / \rho_{ref} / k \hat{T}$, $\hat{T} = \hat{p}/\rho_{ref}$ (since at $T=T_{ref}$ and $n=n_{ref}$ the flow density is $\rho_{ref}$ and the pressure is $p_{ref}$).
Scaling of specific heats:
\begin{equation}
    c_v(T) = \frac{\partial \varepsilon}{\partial T} = \frac{p_{ref}}{\rho_{ref}T_{ref}}\frac{\partial \hat{\varepsilon}}{\partial \hat{T}}=c_{v,ref} \hat{c}_{v}.
\end{equation}
So $c_{v,ref}=p_{ref} / (\rho_{ref}T_{ref}) = k / m_{ref}$. Mayer's relation in scaled form then reads that $\hat{c}_p = \hat{c}_v + 1$.

\section{Navier--Stokes equations}
Let us consider the (multi-temperature) Navier--Stokes equations in general form:
\begin{equation}
     \frac{d}{dt}\rho_s + \rho_s \nabla \cdot \mathbf{v} + \nabla \cdot (\rho_s \mathbf{V}_s) = 0,\quad s=1,\ldots,N_s
\end{equation}

\begin{equation}
     \rho \frac{d}{dt}\mathbf{v} + \nabla \cdot \mathbf{P} = 0,
\end{equation}

\begin{equation}
     \rho \frac{d}{dt}\mathbf{U} + \nabla \cdot \mathbf{q} + \mathbf{P} : \nabla \mathbf{v} = 0,
\end{equation}
\begin{equation}
     \rho \frac{d}{dt}\mathbf{E^v_s} + \nabla \cdot \mathbf{q^v_s} = E_s \nabla \cdot (\rho_s \mathbf{V}_s).
\end{equation}

With scaling:
\begin{equation}
    \frac{\rho_{ref}}{t_{ref}} \frac{d}{d\hat{t}}\hat{\rho}_s + \frac{\rho_{ref} v_{ref}}{L_{ref}} \hat{\rho}_s \nabla_{\hat{\mathbf{x}}} \cdot \hat{\mathbf{v}} + \frac{\rho_{ref}v_{ref}}{L_ref}\nabla_{\hat{\mathbf{x}}} \cdot (\rho_s \hat{\mathbf{V}}_s) = 0,\quad s=1,\ldots,N_s
\end{equation}

\begin{equation}
     \frac{\rho_{ref}v_{ref}}{t_{ref}} \hat{\rho} \frac{d}{d\hat{t}}\hat{\mathbf{v}} + \frac{p_{ref}}{L_{ref}}\nabla_{\hat{\mathbf{x}}} \cdot \hat{\mathbf{P}} = 0,
\end{equation}

\begin{equation}
     \frac{p_{ref}}{t_{ref}} \hat{\rho} \frac{d}{d\hat{t}}\hat{\mathbf{U}} + \frac{q_{ref}}{L_{ref}} \nabla_{\hat{\mathbf{x}}} \cdot \hat{\mathbf{q}} + \frac{p_{ref}v_{ref}}{L_{ref}} \hat{\mathbf{P}} : \nabla_{\hat{\mathbf{x}}} \hat{\mathbf{v}} = 0,
\end{equation}
\begin{equation}
     \frac{p_{ref}}{t_{ref}} \rho \frac{d}{dt}\mathbf{E^v_s} + \frac{q_{ref}}{L_{ref}} \nabla_{\hat{\mathbf{x}}} \cdot \mathbf{q^v_s} = E_s \nabla \cdot (\rho_s \mathbf{V}_s).
\end{equation}

So $q_{ref} = p_{ref}v_{ref}$.

\section{Production terms}


\end{document}